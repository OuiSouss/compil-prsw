\documentclass[11pt,a4paper]{article}
\usepackage[T1]{fontenc}
\usepackage[utf8]{inputenc}
\usepackage[french]{babel}


\title{Rapport Projet de Compilation}
\author{Alexis PICHON, Amélie RISI, Clément SIBILLE, Haize WEISS}
\date{\today}

\begin{document}

\maketitle
\tableofcontents
\pagebreak

\setlength{\parskip}{1em}
\setlength{\intextsep}{2em}


\section{Introduction}
Le but de ce projet était de réaliser, en groupe, un analyseur syntaxique du langage Pseudo-Pascal, un analyseur sémantique, un interpréteur, un compilateur de Pseudo-Pascal vers C3A et enfin un interpréteur C3A.
\par
Dans les parties qui suivent nous allons détailler ce qui a été fait et comment nous avons procédé, avant de traiter des différents problèmes rencontrés lors du projet.




\section{Analyse syntaxique}
Pour l'analyse syntaxique, nous avons repris le travail réalisé lors du TD concernant les tableaux. 

\subsection{La grammaire}
Nous avons ensuite modifié le lexer et le fichier Bison pour obtenir la grammaire voulue. Nos conventions d'écriture sont les suivantes, les non-terminaux sont écrits en lettres minuscules et les ternimaux exclusivement en lettres majsucules. 
La grammaire, fournie initialement dans le sujet de projet, était ambiguë. Pour pallier à ce problème, le non-terminal E de la grammaire initiale a été transformé en expr, term et fact des non-terminaux qui qualifient respectivement l'expression, le terme et le facteur. 
\subsubsection{L'expression}
L'expression est le dernier échellon de priorité des opérateurs donc elle englobe tous les opérateurs qui ont une faible priorité : Plus, Moins, Ou, Inférieur strictement, Egal. 
Ce sont des terminaux écrits respectivement de cette manière dans notre grammaire : PL, MO, OR, EQ.
\subsubsection{Le terme}
Le terme répresente une plus forte priorité par rapport aux précédents. Il correspond aux opérateurs suivants : Multiplication, Et, Non.
Ces opérateurs sont écrits de telles manières dans notre parser : MU, AND, NOT.  
\subsubsection{Le facteur}
Le facteur représente une expression parenthésée, un entier, un indentificateur de variable, un booléen, une déclaration de fonction, une déclaration de tableaux.


\section{Analyse sémantique}



\section{Interpréteur PP}



\section{Compilateur PP vers C3A}
Pour traduire un programme écrit en langage PP vers le langage C3A, nous sommes parti sur la base de la correction du compilateur IMP vers C3A réalisé lors du mini-projet.


\section{Interpréteur C3A}


\section{Problèmes rencontrés}


\section{Conclusion}





\end{document}
