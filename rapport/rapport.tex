\documentclass[11pt,a4paper]{article}
\usepackage[utf8]{inputenc}
\usepackage[french]{babel}


\title{Rapport Projet de Compilation}
\author{Alexis PICHON, Amélie RISI, Clément SIBILLE, Haize WEISS}
\date{\today}

\begin{document}

\maketitle
\tableofcontents
\pagebreak

\setlength{\parskip}{1em}
\setlength{\intextsep}{2em}


\section{Introduction}
Le but de ce projet était de réaliser, en groupe, un analyseur syntaxique du langage Pseudo-Pascal, un analyseur sémantique, un interpréteur, un compilateur de Pseudo-Pascal vers C3A et enfin un interpréteur C3A.
\par
Dans les parties qui suivent nous allons détailler ce qui a été fait et comment nous l'avons fait, avant de traiter des différents problèmes rencontrés lors du projet.
\section{Analyse syntaxique}
Pour l'analyse syntaxique, nous avons repris le travail réalisé lors du TD concernant les tableaux. Nous avons ensuite modifié le lexer et le fichier Bison pour obtenir la grammaire voulue.
\section{Analyse sémantique}
\section{Interpréteur PP}
\section{Compilateur PP vers C3A}
Pour traduire un programme écrit en langage PP vers le langage C3A, nous sommes parti sur la base de la correction du compilateur IMP vers C3A réalisé lors du mini-projet.
\section{Interpréteur C3A}
\section{Problèmes rencontrés}
\section{Conclusion}





\end{document}